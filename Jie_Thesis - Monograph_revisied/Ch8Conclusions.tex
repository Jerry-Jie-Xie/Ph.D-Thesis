% Ch7Conclusions.tex

\chapter[Conclusion]{Conclusion and future work}
\label{cha:cha8Conclusions}

%===============================
This thesis has addressed frog call classification using both trophy and field recordings. For trophy recordings, a combined feature set using temporal, perceptual and cepstral features is proposed. A novel cepstral feature with good anti-noise performance is proposed using wavelet packet decomposition (WPD). 
To classify multiple simultaneously vocalising frog species in field recordings, two classification frameworks are adopted: multiple-instance multiple-label (MIML) learning and multiple-label (ML) learning. 

Challenges of this thesis lie in designing effective feature extraction algorithms and adopting classification frameworks. Key contributions of this research to the challenges are summarised, and useful avenues of inquiry for improving the methods described in this thesis are explored.

\section{Summary of contributions}
	
In Table~\ref{tab:contribution}, the proposed algorithm of each chapter is listed.

\begin{table}[htb!]
\centering
\caption{The list of algorithms used in this thesis}
\label{tab:contribution}
\resizebox{\textwidth}{!}{
\begin{tabular}{lllllll}
\hline\hline
\textbf{Algorithm ID} & \textbf{Data}                                                           & \textbf{Segmentation}                                                     & \textbf{Fature}         & \textbf{Classifier}                                                            & \textbf{Contribution}                                                   & \textbf{Chapter} \\ \hline
1                     & \begin{tabular}[c]{@{}l@{}}Trophy \\ recordings\end{tabular}          & \begin{tabular}[c]{@{}l@{}}Amplitude frequency\\ information\end{tabular} & Various                 & \begin{tabular}[c]{@{}l@{}}kNN, SVM, \\ RF, and NN\end{tabular}                & \begin{tabular}[c]{@{}l@{}}feature and \\ integration\end{tabular}      & 3                \\ 
2                     & \begin{tabular}[c]{@{}l@{}}Trophy and field \\  recordings\end{tabular} & \begin{tabular}[c]{@{}l@{}}Amplitude frequency\\ information\end{tabular}                                         & AWSCCs                  & kNN and SVM                                                                    & feature                                                                 & 4                \\ 
3                     & Field recordings                                                      & AED                                                                       & Various                 & \begin{tabular}[c]{@{}l@{}}MIML-kNN, \\ MIML-SVM,\\  and MIML-RBF\end{tabular} & \begin{tabular}[c]{@{}l@{}}segmentation \\ and integration\end{tabular} & 5                \\ 
4                     & Field recordings                                                      & No segmentation                                                           & \begin{tabular}[c]{@{}l@{}}LPCs, MFCCs,\\ and AWSCCs \end{tabular} & \begin{tabular}[c]{@{}l@{}}ML-kNN, ML-DT, \\ and ML-RF\end{tabular}            & feature and integration                                                 & 6                \\ \hline\hline
\end{tabular}
}
\end{table}

Detailed contributions of this thesis are summarised below:

%\begin{itemize}
	
\textbf{(1)} \textit{An enhanced acoustic feature set for frog call classification in trophy recordings.} 
\\
Effectively modelling frog vocalisations has significant impact on the performance of frog call classification systems. A novel feature set is proposed to represent frog calls using temporal, perceptual, and cepstral information. A combination of temporal, perceptual, and cepstral features can greatly increase the discriminability of the combined feature set. Evaluations of the propose feature set are based on 24 frog species from trophy recordings. Five machine learning algorithms are compared to the proposed feature set. Background noise with SNR from -10 dB to 40 dB is added to test the anti-noise ability of the proposed feature set. Experimental results show that (1) Compared to previous feature sets, an enhanced feature set including  can achieve the best classification performance. (2) The best classification performance is achieved by SVM and RF, in comparison with LDA, K-NN, and MLP. (3) The cepstral feature is very sensitive to the background noise, but can achieve high classification accuracy for high SNR recordings. 

\textbf{(2)} \textit{A novel feature via adaptive WPD for frog call classification in both trophy and field recordings.}
\\
Cepstral features are widely used for classifying frog calls. Although cepstral features have shown high classification performance for classifying frog species in trophy recordings, the performance is quickly decreased when classifying frog species in field recordings. A novel cepstral feature via WPD is proposed to increase the anti-noise ability. An adaptive frequency scale is generated by applying k-means clustering to all dominant frequencies of training datasets. Compared to other frequency scales, the adaptive frequency scale can better reflect the frequency distribution of frog calls. Evaluations of the propose feature set are based on 18 frog species from trophy recordings and eight frog species from field recordings. Experimental results in both trophy and field recordings show that the propose cepstral feature can achieve the best classification performance when compared to other cepstral features using trophy recordings. For field recordings, the classification performance of our cepstral features does not greatly decrease unit the SNR is 10 dB.



\textbf{(3)}  \textit{Design a MIML classification framework for frog call classification in field recordings.} 
\\
Most field recordings contain multiple simultaneously vocalising frog species, a single-instance single-label (SISL) classification framework might be unfit for classifying frog species in those field recordings. Compared to SISL learning, MIML learning is a natural fit for field recordings of frogs. A novel MIML classification framework is adopted to focus on frog calls. To segment individual frog syllables, a novel AED algorithm is designed based on event filtering. The propose classification framework is evaluated using 342 10-second recordings including eight frog species. Experimental results show that MIML-RBF achieves the best classification results with shape based feature sets. Compared to SISL learning, MIML learning can significantly increase the classification results for all eight frog species.

\textbf{(4)}  \textit{Design a ML classification framework for long-term monitoring of frogs in field recordings.} 
\\ 
Compared to SISL learning, MIML learning shows a better performance for classifying multiple simultaneously vocalising frog species in field recordings.
However, MIML classification performance is highly affected by the AED method.
To reduce the effect of AED, a novel ML classification framework is adopted.
A new method for constructing cepstral features is proposed. Evaluations of the ML classification framework are based on the same dataset with MIML learning. The ML classification performance is slightly better than MIML learning, but non-use of the segmentation process can greatly increase the classification efficiency.


%In summary, the contributions of this thesis mainly include three folds.
%
%\begin{enumerate}
%\item Syllable segmentation: Segmenting individual frog syllables from the background noise.
%
%\item Feature representation: Identifying characteristics of Australian frog vocalisations and developing novel feature extraction algorithms.
%
%\item System integration: Designing novel classification frameworks that combine signal processing and machine learning algorithms into a unified and effective system to classify frog vocalisations.

%\end{enumerate}






\section{Limitations and future work}
Although our proposed frog call classification framework shows promising classification performance, there is still much work that can be done to help scientists and researchers in data collection and analysis of the bio-acoustics communities.


\begin{itemize}
%\item  One of the most important issues when dealing with frog recordings is the need for the standardised species-specific data with behavioural labels. The algorithms we developed for frog call classification can thus be evaluated on a larger dataset. Researchers can also use the outcomes of such automatic call classification methods for field studies effectively and precisely. However, it is very time-consuming to perform manual labelling. It is necessary to develop automatic or semi-automatic methods to perform the labelling.

\item For one frog species, calling parameters of different areas might have some variations. It is necessary to investigate our proposed classification framework for classifying frog vocalisations from different areas.

\item Since field recordings often contain much background noise, it is important to develop effective noise reduction algorithms to reduce the background noise and improve the classification performance.


\item For each frog, there are many types of frog calls: (1) mating calls, (2) territorial calls, (3) male release calls, (4) female release calls, (5) distress calls, and (6) warning calls. Among them, almost all studies that use machine learning algorithm to classify frog calls select mating calls (advertisement calls) as the research targets. It it worthwhile to classify frog calls into different species and further classify different types of calls for one frog species.



\item  One aspect that requires further improvement is the need for an advanced frog syllable segmentation method for the field recordings so as to extract more accurate event-based features and conduct more thorough analysis on frog vocalisations. The problem of syllable segmentation is very complicated, because field recordings often have many simultaneous overlapping calling activities from birds, frogs, insects, and many other sources. 


\item Our developed frog call classification framework aims to help ecologists to study frogs over larger spatial and temporal scales. However, there is still no a generic platform for running frog recordings. It is necessary to develop a toolbox with an easy user interface for frog call classification, and then ecologists can conduct the analysis on their own. We focus on efficacy in this research, however efficiency is also very important in big data analysis. For this purpose, the MATLAB code corresponding to feature extractors and classifiers needs to be optimised to perform real-time frog call classification in the field.


\end{itemize}










