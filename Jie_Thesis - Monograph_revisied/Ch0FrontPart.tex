% Part0FrontPart.tex

%============frontmatter % Coverpage etc
\title{Acoustic classification of Australian frogs for ecosystem surveys}
       
\author{Jie Xie}
%\authoremail{xiej8734@gmail.com}
\supervisor{Dr Jinglan Zhang, Professor Paul Roe, Dr Michael Towsey, Professor Vinold Chandran}
\thesistype{Doctor of Philosophy}    % or comment it out (the default is Doctor of Philosophy)
\university{Queensland University of Technology} % or comment it out (the default is QUT)
\faculty{Science and Engineering Faculty}   % or comment it out (the default one is SEF)
\school{School of Electrical Engineering and Computer Science}   % or School of xxx xxx xxx 
                         % or comment it out (the default one is Elec Eng and Computer Science)                         
%\universitylogo{yes}{1.0}{./QUTLogo.eps}  % yes (true, 1) or no (false, 0), scale = 1.0; filename = QUTLogo.eps

%

\submissiondate{July 2016}
\copyrightyear{2016}
%\informationcutoffdate{01 March 2010} %for cut of date of information in the thesis

\maketitle	   %cover page of the thesis; a blank page is automatically added for double-side printing

%\blankpage         %two more blank pages. If you don't want them, comment them out
% \blankpage

\setcounter{page}{1} %start to count page numbers

% \insidetitlepage   %if you don't like an insidetitle page, comment it out

\copyrightpage     %another format available: \copyrightpageWithTitle. You may try it (with thesis title). 

%\signaturepage\cleardoublepage   % you may not need this signature page, so comment it out

%============dedication
\begin{dedication}
To my family 
\end{dedication}

%============abstract
\begin{abstract}

A rapid decrease in frog populations, which is regarded as one of the most critical threats to global biodiversity, has been spotted from locations around the world. Causes of this decline can be summarised as follows: disease, habitat destruction and modification, exploitation, pollution, and introduced species. Since frogs play an important role in the whole ecosystem, the decline of frog populations necessitates the monitoring of frogs. For some frog species, a small body size makes them difficult to be found in the field by the visual inspection. Acoustic monitoring of frogs is thus more widely used. To collect frog vocalisations, the traditional method requires ecologists and volunteers to visit the field, which limits the scale for acoustic data collection. 
In contrast, recent advances in acoustic sensors provide a novel method to survey vocalising animals such as frogs. Once acoustic sensors are successfully installed in the field, acoustic data can be automatically collected at large spatial and temporal scales. For each acoustic sensor, 
several gigabytes of compressed audio data can be generated per day, and large volumes of raw acoustic data have been collected. Enabling automated species identification in acoustic data has become very important to gain insights about frogs and the environment.
For those acoustic data collected using acoustic sensors, the desired signal (frog call) is often very weak and there are multiple signals overlapping the frog calls. All these characteristics pose a big challenge to performing automatic classification of frog species in acoustic data.



This dissertation aims to build a robust and accurate frog species classification framework in acoustic data, of which two important aspects are investigated: feature extraction and classification. The contributions of this dissertation can be summarised as follows.
\begin{enumerate} 
\item[(1)]	Develop a combined feature set for frog call classification in high SNR recordings (Chapter \ref{cha:cha4EnhancedFeature}). 
\\
Time-frequency information of frog calls can be effectively represented via a combination of temporal, perceptual, and cepstral features. 
The classification performance of various machine learning techniques is compared with different feature sets. Our proposed combined feature set achieves a satisfied classification accuracy.
 
\item[(2)]	Propose a novel cepstral feature based on adaptive WPD (Chapter \ref{cha:cha5WaveletFeature}). 
\\
To better capture the frequency domain information of frog calls with a good anti-noise ability, a novel feature, namely \textit{adaptive frequency scaled wavelet packet decomposition sub-band cepstral coefficients}, is proposed. Compared with other cepstral coefficients, the proposed feature shows the best classification performance and a good anti-noise ability.

\item[(3)]	Design a MIML framework to classify multiple simultaneously vocalising frog species in low SNR recordings (Chapter \ref{cha:cha6MIML}).
\\
To use MIML learning for classifying multiple simultaneously vocalising frog species, AED is first used to segment frog syllables. Then, various features are calculated based on the content and shape of each segmented syllable. Next, a bag generator is applied to those extracted features for a bag-level feature representation. Finally, three MIML classifiers are employed for the classification of frog vocalisations: MIML-SVM, MIML-KNN, and MIML-RBF. 

\item[(4)]	Design a ML framework to classify multiple simultaneously vocalising frog species in low SNR recordings (Chapter \ref{cha:cha7ML}).
\\
For the ML learning, global features are first calculated without segmentation. Then, ML learning is used to classify multiple simultaneously vocalising frog species using global features. Three ML classifiers are compared: Binary relevance, Classifier Chains, Random k-labelsets, where the base classifier is a decision tree classifier. \end{enumerate}




Our proposed approach achieves promising classification results compared with most previous studies. Novel feature sets and classification learning frameworks have different contributions to the performance of the classification system of frog vocalisations. To cope with high SNR recordings, we construct a novel feature set including temporal, perceptual, and cepstral features. To improve the anti-noise ability of cepstral features, we develop a novel wavelet-based ceptral feature. To address low SNR recordings with multiple overlapping vocalising frog species, the classification frameworks of MIML learning and ML learning are proposed. To the best of this researcher's knowledge, it is the first time that MIML learning and ML learning are employed for automatic classification of multiple simultaneously vocalising frog species.
With this developed classification system, the ecosystem at large spatial and temporal scales can be surveyed, which can help ecologists better understand the ecosystem. 


\end{abstract}



%\newpage
%\begin{center}
%{\huge \textbf{List of Abbreviations}}
%\end{center}
%
%\begin{table}[htb!]
%%\caption{My caption}
%%\label{my-label}
%\begin{tabular}{lllll}
%DFT   &  &  &  & Discrete Fourier Transform          \\
%DCT   &  &  &  & Discrete Cosine Transform           \\
%SNR   &  &  &  & Signal to Noise Ratio               \\
%LPCs  &  &  &  & Linear Predictive Coding            \\
%MFCCs &  &  &  & Mel-Frequency Cepstral Coefficients \\
%LDA   &  &  &  & Linear Discriminant Analysis        \\
%K-NN  &  &  &  & K-Nearest Neighbour                 \\
%SVM   &  &  &  & Support Vector Machine              \\
%ANN   &  &  &  & Artificial Neural Network           \\
%RF    &  &  &  & Random Forest                       \\
%AED   &  &  &  & Acoustic Event Detection            \\
%WPD   &  &  &  & Wavelet Packet Decomposition        \\
%MIML  &  &  &  & Multiple-Instance Multiple-Label    \\
%ML    &  &  &  & Multiple Label                     
%\end{tabular}
%\end{table}



%============keywords
\begin{keywords}
Bioacoustic monitoring \\
Soundscape ecology\\
Environmental audio analysis \\
Frog call classification \\
Spectrogram analysis \\
Acoustic feature extraction \\
Wavelet packet decomposition \\
Multiple-instance multiple-label learning \\
Multiple-label learning \\
 
\end{keywords}






%============acknowledgement
\begin{ack}
First, I would like to express my sincere gratitude and thanks to Dr Jinglan Zhang (principal supervisor), for giving me an opportunity to study in Australia. During the entirety of this PhD study, I have learnt so much from her about having passion for work, combined with high motivation, which will benefit me throughout my life. 
I would also like to express my gratitude to Professor Paul Roe (associate supervisor), for his consistent instructions and supports through the last three years.  

I would also like to thank Dr Michael Towsey for his provision of consistent guidance, discussions, and encouragement during my PhD study. Michael's attitude towards scientific research keeps motivating me go deeper into research.  


I want to thank Professor Vinod Chandran for his support in writing my confirmation report and this thesis. Vinod's strong background knowledge in signal processing greatly helps me improve my understanding of this research.

I would also like to express my gratitude to my family, especially my grandparents, parents and my wife. They have always supported my overseas study. Without their support, I could not give my full attention to PhD study and  the completion of this thesis. 
My sincere thanks also go to all my friends for their love, attention and support to my PhD study. 

Finally, I extend my thanks to the China Scholarship Council (CSC), Queensland University of Technology and the Wet Tropics Management Authority for their financial support. 

\end{ack}


%=============preface
%\begin{preface}
%%Here is the preface. 
%%
%\end{preface}


%==============
% If you don't like to put nomenclature, which you have manually 
% edited in the file nomenclature.tex, at the front, comment the 
% following line out
% \listnomenclatureatfront{yes}{./nomenclature.tex} 

%==============
\afterpreface
