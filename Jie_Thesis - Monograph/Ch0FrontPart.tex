% Part0FrontPart.tex

%============frontmatter % Coverpage etc
\title{Acoustic classification of Australian frogs for ecosystem surveys}
       
\author{Jie Xie}
%\authoremail{xiej8734@gmail.com}
\supervisor{Dr Jinglan Zhang, Professor Paul Roe, Dr Michael Towsey, Professor Vinold Chandran}
\thesistype{Doctor of Philosophy}    % or comment it out (the default is Doctor of Philosophy)
\university{Queensland University of Technology} % or comment it out (the default is QUT)
\faculty{Science and Engineering Faculty}   % or comment it out (the default one is SEF)
\school{School of Electrical Engineering and Computer Science}   % or School of xxx xxx xxx 
                         % or comment it out (the default one is Elec Eng and Computer Science)                         
%\universitylogo{yes}{1.0}{./QUTLogo.eps}  % yes (true, 1) or no (false, 0), scale = 1.0; filename = QUTLogo.eps

%

\submissiondate{April 2016}
\copyrightyear{2016}
%\informationcutoffdate{01 March 2010} %for cut of date of information in the thesis

\maketitle	   %cover page of the thesis; a blank page is automatically added for double-side printing

%\blankpage         %two more blank pages. If you don't want them, comment them out
% \blankpage

\setcounter{page}{1} %start to count page numbers

% \insidetitlepage   %if you don't like an insidetitle page, comment it out

\copyrightpage     %another format available: \copyrightpageWithTitle. You may try it (with thesis title). 

%\signaturepage\cleardoublepage   % you may not need this signature page, so comment it out

%============dedication
\begin{dedication}
To my family 
\end{dedication}

%============abstract
\begin{abstract}
Rapid decreases in frog populations have been spotted from locations around the world, which are regarded as one of the most critical threats to the global biodiversity. Causes of these declines can be summarised as follows: disease, habitat destruction and modification, exploitation, pollution, pesticide use, introduced species, and ultraviolet-B radiation (UV-B). On the one hand frogs play an important role in the whole ecosystem, but on the other frog populations are declining globally. To assess frog populations and optimise the protection policy, monitoring frogs is becoming ever more necessary. Since frogs are much easier to be heard than seen, frog populations are often assessed by their vocalisations. In order to collect frog vocalisations, traditional manual methods require ecologists and volunteers to visit the field, which limits the        
scale for acoustic data collection. In contrast, recent advances in acoustic sensors provide a novel method to survey vocalising animals such as frogs. After deploying acoustic sensors in the field, they can automatically collect acoustic data at large spatial and temporal scales. The large volumes of raw acoustic data collected must be analysed to gain insights about frogs and the environment from the data It has become very important to enable automated species identification in acoustic data. Also since the data is collected from the field, the acoustic data tend to be very noisy and very often the desired signal (frog call) is weak. There are also very often multiple overlapping signals over the frog calls. These characteristics pose a big challenge to perform the automatic classification of frog species in acoustic data.



The research presented in this dissertation aims to investigate methods to build a robust and high performance classification system for frog species in acoustic data. Two important aspects of a classification system are investigated: feature extraction and classification, which consist of contributions towards three main objectives:
\begin{enumerate} 
\item[(1)]	Develop an enhanced feature representation for frog call classification. Time-frequency information of frog calls can be effectively represented via the enhanced representation of temporal, perceptual and cepstral features. The classification performance of various machine learning techniques is compared with different feature representations. Our proposed enhanced feature representation achieves the best classification accuracy which outperforms most previous studies.
 
\item[(2)]	Propose a novel feature representation based on adaptive wavelet packet decomposition. To better capture the frequency domain information of frog calls with a good anti-noise ability, a novel feature representation is proposed named \textit{adaptive frequency scaled wavelet packet decomposition sub-band cepstral coefficients}. Compared with other cepstral coefficients, our proposed feature representation shows the best classification performance and a good anti-noise ability.

\item[(3)]	Design a robust classification system to study low signal-to-noise ratio (SNR) recordings with multiple simultaneously vocalising frog species. Two classification frameworks are employed to classify multiple simultaneously vocalising frog species. 


\begin{enumerate}
\item Multiple-instance Multiple-label (MIML) learning
\\
To use MIML learning for classifying multiple simultaneously vocalising frog species, each individual syllables are first segmented. Then, various features are calculated from each segmented syllable. Next, a bag generator is applied to those extracted features to construct a suitable bag-of-syllable representation. Finally, three MIML learning algorithms are employed for the classification of frog vocalisations: MIML-SVM, MIML-KNN, and MIML-RBF. 

\item Multiple-label (ML) learning
\\
As for the ML learning, acoustic features are first calculated without segmentation. Then, ML learning is used to classify simultaneously vocalising frog species using extracted features. Three main ML learning methods are compared: Binary relevance, Classifier Chains, Random k-labelsets, where the base classifier is decision tree. Furthermore, the frog abundance and species richness over three months are calculated based on the results of acoustic event detection and ML classification, respectively. 
Lastly, the correlation analysis between frog calling activity (frog abundance and species richness) and weather variables (mean temperature and rainfall) are studied to demonstrate the application of our proposed ML classification framework.
\end{enumerate}


\end{enumerate} 


Our proposed approach achieves promising classification results compared with most previous studies. Novel feature representations and classification learning frameworks have different contributions to the performance of the classification system of frog vocalisations. To cope with high SNR recordings, we construct a novel feature representation including temporal, perceptual, and cepstral features. To improve the anti-noise ability of cepstral features, we develop a novel wavelet-based ceptral feature representation. To address low SNR recordings with multiple overlapping vocalising frog species, the classification framework of MIML learning and ML learning are proposed. To the best of our knowledge, it is the first time that MIML learning and ML learning are employed for automatic classification of multiple simultaneously vocalising frog species.
With our developed classification system, we can survey the ecosystem at large scale and temporal scales, which can help ecologists better understand the ecosystem. 


\end{abstract}



%\newpage
%\begin{center}
%{\huge \textbf{List of Abbreviations}}
%\end{center}
%
%\begin{table}[htb!]
%%\caption{My caption}
%%\label{my-label}
%\begin{tabular}{lllll}
%DFT   &  &  &  & Discrete Fourier Transform          \\
%DCT   &  &  &  & Discrete Cosine Transform           \\
%SNR   &  &  &  & Signal to Noise Ratio               \\
%LPCs  &  &  &  & Linear Predictive Coding            \\
%MFCCs &  &  &  & Mel-Frequency Cepstral Coefficients \\
%LDA   &  &  &  & Linear Discriminant Analysis        \\
%K-NN  &  &  &  & K-Nearest Neighbour                 \\
%SVM   &  &  &  & Support Vector Machine              \\
%ANN   &  &  &  & Artificial Neural Network           \\
%RF    &  &  &  & Random Forest                       \\
%AED   &  &  &  & Acoustic Event Detection            \\
%WPD   &  &  &  & Wavelet Packet Decomposition        \\
%MIML  &  &  &  & Multiple-Instance Multiple-Label    \\
%ML    &  &  &  & Multiple Label                     
%\end{tabular}
%\end{table}





\newpage
\begin{center}
{\huge \textbf{List of Publications}}
\end{center}

{ \large \textbf{Journal Article}}
\begin{enumerate} 
\item	\textbf{Xie, Jie}, Towsey, Michael, Zhang, Jinglan, and Roe, Paul (2016) Adaptive frequency scaled wavelet packet decomposition for frog call classification.  Ecological Informatics, 32, pp. 134-144.
\item	Zhang Liang, Towsey Michael, \textbf{Xie Jie}, Zhang Jinglan, Roe Paul,  Using multi-label classification for acoustic pattern detection and assisting bird species surveys, Applied Acoustics, Volume 110, September 2016, Pages 91-98.
\item \textbf{Xie, Jie}, Towsey, Michael, Zhang, Jinglan, and Roe, Paul, Frog call classification: a survey, Artificial Intelligence Review (\textbf{Under review})

\item  \textbf{Xie, Jie}, Towsey, Michael, Zhang, Jinglan, and Roe, Paul, Frog call classification based on enhanced features and machine learning algorithms, Applied acoustics (\textbf{Under review})


\end{enumerate}

{ \large \textbf{Conference Paper}}
\begin{enumerate} 

\item \textbf{Xie, Jie}, Towsey, Michael, Zhang, Jinglan, and Roe, Paul, Detecting frog calling activity based on acoustic event detection and multi-label learning, International Conference on Computational Science (Accepted)
 
 
\item \textbf{Xie, Jie}, Towsey, Michael,  Zhang, Liang, Zhang, Jinglan, and Roe, Paul, Multiple-Instance Multiple-Label Learning for the Classification of Frog Calls With Acoustic Event Detection,  International Conference on Image and Signal Processing (Accepted)
 
 
\item \textbf{Xie, Jie}, Towsey, Michael, Zhang, Liang, Zhang, Jinglan, and Roe, Paul, Feature Extraction Based on Bandpass Filtering for Frog Call Classification, International Conference on Image and Signal Processing (Accepted)



\item	\textbf{Xie, Jie}, Towsey, Michael, Truskinger, Anthony, Eichinski, Philip, Zhang, Jinglan, and Roe, Paul (2015) Acoustic classification of Australian anurans using syllable features. In 2015 IEEE Tenth International Conference on Intelligent Sensors, Sensor Networks and Information Processing (ISSNIP), IEEE, Singapore, pp. 1-6.

\item	\textbf{Xie, Jie}, Towsey, Michael, Yasumiba, Kiyomi, Zhang, Jinglan, and Roe, Paul (2015) Detection of anuran calling activity in long field recordings for bio-acoustic monitoring. In 2015 IEEE Tenth International Conference on Intelligent Sensors, Sensor Networks and Information Processing (ISSNIP), IEEE, Singapore, pp. 1-6.
\item	\textbf{Xie, Jie}, Towsey, Michael, Zhang, Jinglan, and Roe, Paul (2015) Image processing and classification procedure for the analysis of Australian frog vocalisations. InProceedings of the 2nd International Workshop on Environmental Multimedia Retrieval, ACM, Shanghai, China, pp. 15-20.
\item	\textbf{Xie, Jie}, Towsey, Michael, Zhang, Jinglan, Dong, Xueyan, and Roe, Paul (2015)Application of image processing techniques for frog call classification. In IEEE International Conference on Image Processing (ICIP 2015), 27-30 September 2015, Québec City, Canada.

\item	\textbf{Xie, Jie}, Towsey, Michael, Eichinski, Philip, Zhang, Jinglan, and Roe, Paul (2015)Acoustic feature extraction using perceptual wavelet packet decomposition for frog call classification. In 2015 IEEE 11th International Conference on e-Science (e-Science), IEEE, Munich, Germany, pp. 237-242.

\item	\textbf{Xie, Jie}, Zhang, Jinglan and Roe, Paul,  “Discovering acoustic feature extraction and selection algorithms for frog vocalization monitoring with machine learning techniques”, 2015 Annual Conference of the Ecological Society of Australia. (Abstract accepted for poster presentation) 

\item	\textbf{Xie, Jie}, Zhang, Jinglan, and Roe, Paul (2015) Acoustic features for hierarchical classification of Australian frog calls. In 10th International Conference on Information, Communications and Signal Processing, 2-4 December 2015, Singapore.

\item	Dong, Xueyan, \textbf{Xie, Jie}, Towsey, Michael, Zhang, Jinglan, and Roe, Paul (2015)Generalised features for bird vocalisation retrieval in acoustic recordings. In IEEE International Workshop on Multimedia Signal Processing, 19-21 October 2015, Xiamen, China.

\end{enumerate} 



%============keywords
\begin{keywords}
Bio-acoustic monitoring \\
Environmental audio analysis \\
Frog call classification \\
Spectrogram analysis \\
Acoustic feature extraction \\
Wavelet packet decomposition \\
Multiple-instance multiple-label learning \\
Multiple-label learning \\
 
\end{keywords}

%============acknowledgment
\begin{ack}
First, I would like to express my sincere gratitude and thanks to Dr Jinglan Zhang (principal supervisor). I want to thank Jinglan for giving me an opportunity to study in Australia. During the whole PhD study, I learnt so much from her about passion for work and high motivation, which will benefit me throughout my life. 
I would also like to express my gratitude to Professor Paul Roe (associate supervisor), for his consistent instructions and supports through the last three years.  

I would also like to thank Dr Michael Towsey for his provision of consistent guidance, discussions, and encouragement during my Phd study. Michael's attitude towards the scientific research keeps motivating me go deep into the research.  


I want to thank Professor Vinod Chandran for his support in writing my confirmation report and this thesis. Vinod's strong background knowledge in signal processing greatly helps me improve my understanding of this research.

I would also like to express my gratefulness to my family, especially my grandparents, parents and my wife. They always support my oversea study silently and firmly. Without their support, I could not pay full attention to PhD study and complete this thesis. 
My sincere thanks also go to all the friends for their love, attention and continuous concern about my PhD study. 

I also want to thank the China Scholarship Council (CSC), Queensland University of Technology and the Wet Tropics management authority for their financial support. 

\end{ack}


%=============preface
%\begin{preface}
%%Here is the preface. 
%%
%\end{preface}


%==============
% If you don't like to put nomenclature, which you have manually 
% edited in the file nomenclature.tex, at the front, comment the 
% following line out
% \listnomenclatureatfront{yes}{./nomenclature.tex} 

%==============
\afterpreface
